%%%%%%%%%%%%%%%%%%%%%%%%%%%%%%%%%%%%%%%%%%%%%%%%%%%%%%%%%%%%%%%%%%%%%%%%%%%%%%%%%%%%%%%%
%%%%%%%%%%%%%                       PREAMBULO                         
%%%%%%%%%%%% Criado por: Washington S. Silva
%%%%%%%%%%%% Em: 21/12/2020
%%%%%%%%%%%%%%%%%%%%%%%%%%%%%%%%%%%%%%%%%%%%%%%%%%%%%%%%%%%%%%%%%%%%%%%%%%%%%%%%%%%%%%%%

\documentclass[a4paper,12pt]{report}
\usepackage[brazil]{babel}
\usepackage[utf8]{inputenc}
\usepackage{ae}
\usepackage[nobib]{CoverPage}
\usepackage[top=2.5cm,left=3cm,right=2.5cm,bottom=2.5cm]{geometry}
\usepackage{enumerate}
\usepackage{caption}
\usepackage{color}
\usepackage[running,left]{lineno}
\usepackage[small,center]{titlesec}
\usepackage{hyperref}
\hypersetup{
	colorlinks = true,
	linkcolor = blue,    
	urlcolor = blue,
}
\usepackage{fancyhdr}
\pagestyle{fancy}

\lhead{}

\chead{Regimento Interno - Diretoria de Ensino}

\rhead{}

\lfoot{Versão de 21/12/2020}

\rfoot{IFMG - Campus Formiga}

\cfoot{\thepage}
\linespread{1.3}
\newcommand{\ORD}[2]{#1\raise1ex\hbox{\scriptsize#2}}
\renewcommand\thechapter{\Roman{chapter}}
\setcounter{secnumdepth}{0}
\setlength\parindent{0pt}

%%%%%%%%%%%%%%%%%%%%%%%%%%%%%%%%%%%%%%%%%%%%%%%%%%%%%%%%%%%%%%%%%%%%%%%%%%%%%%%%%%%%%%%%%%
%%%%%%                               CAPA                                            %%%%%
%%%%%%%%%%%%%%%%%%%%%%%%%%%%%%%%%%%%%%%%%%%%%%%%%%%%%%%%%%%%%%%%%%%%%%%%%%%%%%%%%%%%%%%%%%
\renewcommand{\CPInstituteFont}{\large\sffamily\mdseries\upshape}
\renewcommand{\CoverPageHeader}{\centerline{\Huge{\textbf{Regimento Interno: Minuta}}}}
\renewcommand{\CoverPageFooterLogo}{}
\renewcommand{\CoverPageFooterInfo}{}

  \CoverPageSetup
                 {title = {Diretoria de Ensino \\ IFMG - Campus Formiga}}

  \CoverPageSetup
                 {author = {}}

  \CoverPageSetup
                 {institute = {
                   Rua São Luiz Gonzaga, s/n,                                      \\
                   São Luiz - Formiga/MG - CEP: 35.577-010 - Brasil}}

  \CoverPageSetup
                 {insource = {https://www.formiga.ifmg.edu.br/}}


  \CoverPageSetup
                 {copyright = Versão de 21/12/2020}
%%%%%%%%%%%%%%%%%%%%%%%%%%%%%%%%%%%%%%%%%%%%%%%%%%%%%%%%%%%%%%%%%%%%%%%%%%%%%%%%%%%%%%%%%%
%%%%%%%%%%%%%%%%%%%%%%%%%%%%%%%%%%%%%%%%%%%%%%%%%%%%%%%%%%%%%%%%%%%%%%%%%%%%%%%%%%%%%%%%%%
\begin{document}

%\linenumbers
\tableofcontents

\renewcommand*{\chaptername}{TÍTULO}
%\renewcommand*{\sectionname}{CAPÍTULO}
%%%%%%%%%%%%%%%%%%%%%%%%%%%%%%%%%%%%%%%%%%%%%%%%%%%%%%%%%%%%%%%%%%%%%%%%%%%%%%%%%%%%%%%%%%
%%%%%%                    TITULO I - DA DIRETORIA DE ENSINO                         %%%%%% 
%%%%%%%%%%%%%%%%%%%%%%%%%%%%%%%%%%%%%%%%%%%%%%%%%%%%%%%%%%%%%%%%%%%%%%%%%%%%%%%%%%%%%%%%%%
\chapter{DA DIRETORIA DE ENSINO}
\setcounter{page}{1}
\pagenumbering{arabic}

%%%%%%%%%%%%%%%%%%%%%%%%%%%%%%%%%%%%%%%%%%%%%%%%%%%%%%%%%%%%%%%%%%%%%%%%%%%%%%%%%%%%%%%%%%
%%%%%%                  CAPÍTULO I DAS DISPOSIÇÕES PRELIMINARES                     %%%%%% 
%%%%%%%%%%%%%%%%%%%%%%%%%%%%%%%%%%%%%%%%%%%%%%%%%%%%%%%%%%%%%%%%%%%%%%%%%%%%%%%%%%%%%%%%%%

\section{CAPÍTULO I \\ DAS DISPOSIÇÕES PRELIMINARES}

Art.~\ORD{1}{o} O presente Regimento Interno dispõe sobre a Diretoria de Ensino do IFMG 
Campus Formiga, sua missão, finalidade, sua organização, bem como suas respectivas 
competências.

%%%%%%%%%%%%%%%%%%%%%%%%%%%%%%%%%%%%%%%%%%%%%%%%%%%%%%%%%%%%%%%%%%%%%%%%%%%%%%%%%%%%%%%%%%
%%%%%%                     CAPÍTULO II DA MISSÃO                                    %%%%%%
%%%%%%%%%%%%%%%%%%%%%%%%%%%%%%%%%%%%%%%%%%%%%%%%%%%%%%%%%%%%%%%%%%%%%%%%%%%%%%%%%%%%%%%%%%

\section{CAPÍTULO II \\ DA MISSÃO}

Art.~\ORD{2}{o} Viabilizar o bom funcionamento dos cursos técnicos e de graduação e 
promover a melhoria contínua dos processos administrativos e de suporte ao ensino.

%%%%%%%%%%%%%%%%%%%%%%%%%%%%%%%%%%%%%%%%%%%%%%%%%%%%%%%%%%%%%%%%%%%%%%%%%%%%%%%%%%%%%%%%%%
%%%%%%                     CAPÍTULO III DA FINALIDADE                               %%%%%%
%%%%%%%%%%%%%%%%%%%%%%%%%%%%%%%%%%%%%%%%%%%%%%%%%%%%%%%%%%%%%%%%%%%%%%%%%%%%%%%%%%%%%%%%%%

\section{CAPÍTULO III \\ DA FINALIDADE}

Art.~\ORD{3}{o} A Diretoria de Ensino, conduzida por um(a) diretor(a) nomeado pelo 
Diretor-Geral, é o órgão executivo que planeja, coordena, fomenta, executa e supervisiona 
as atividades e políticas institucionais de ensino, articuladas à pesquisa e à extensão.

%%%%%%%%%%%%%%%%%%%%%%%%%%%%%%%%%%%%%%%%%%%%%%%%%%%%%%%%%%%%%%%%%%%%%%%%%%%%%%%%%%%%%%%%%%
%%%%%%                     CAPÍTULO V DA ORGANIZAÇÃO                                %%%%%% 
%%%%%%%%%%%%%%%%%%%%%%%%%%%%%%%%%%%%%%%%%%%%%%%%%%%%%%%%%%%%%%%%%%%%%%%%%%%%%%%%%%%%%%%%%%

\section{CAPÍTULO IV \\ DA ORGANIZAÇÃO}

Art.~\ORD{4}{o} Constituem setores administrativos da Diretoria de Ensino:

\begin{enumerate}
	\renewcommand{\labelenumi}{\Roman{enumi}}
	
	\item Diretoria de Ensino
	
	\begin{enumerate}
		\item[a)] Setor de Registro e Controle Acadêmico;
		\item[b)] Seção de Planejamento de Ensino de Graduação;
		\item[c)] Seção de Planejamento de Ensino;
		\item[d)] Seção Pedagógica;
		\item[e)] Seção de Laboratórios;
		\item[f)] Seção de Assuntos Estudantis;
	\end{enumerate}   
\end{enumerate}

%%%%%%%%%%%%%%%%%%%%%%%%%%%%%%%%%%%%%%%%%%%%%%%%%%%%%%%%%%%%%%%%%%%%%%%%%%%%%%%%%%%%%%%%%%
%%%%%%           CAPÍTULO VI DAS COMPETÊNCIAS DA DIRETORIA DE ENSINO                 %%%%%
%%%%%%%%%%%%%%%%%%%%%%%%%%%%%%%%%%%%%%%%%%%%%%%%%%%%%%%%%%%%%%%%%%%%%%%%%%%%%%%%%%%%%%%%%%

\section{CAPÍTULO V \\ DAS COMPETÊNCIAS DA DIRETORIA DE ENSINO}

Art.~\ORD{5}{o} A Diretoria de Ensino é o orgão responsável por planejar, organizar e 
direcionar as ações relativas ao ensino, sendo subordinada à Diretoria-Geral do Campus, 
ao qual compete:

\begin{enumerate}
\renewcommand{\labelenumi}{\Roman{enumi}}

\item supervisionar e implementar as políticas institucionais de ensino do IFMG;

\item gerenciar as atividades e serviços de apoio ao ensino;

\item assegurar a observância dos Regulamentos de Ensino, dos projetos pedagógicos dos 
     cursos e das resoluções dos órgãos superiores;

\item elaborar cronograma anual de ações internas prioritárias para o bom andamento do 
      ensino;

\item emitir pareceres, elaborar minutas, instruções e demais documentos necessários ao 
      andamento do ensino;

\item propor os calendários acadêmicos anuais de referência do Campus;

\item conduzir o processo de elaboração dos horários oficiais de aulas e de recuperação;

\item gerenciar e promover os programas de monitoria e tutorias;

\item responsabilizar-se pela guarda de documentos inerentes ao setor, considerando os 
      prazos legais;

\item propor aos órgãos competentes a adoção de medidas necessárias à estruturação 
      curricular dos cursos em seus aspectos legais e formais, ao aperfeiçoamento 
      da administração acadêmica e à melhoria das condições materiais do ensino;

\item supervisionar e assessorar os coordenadores de cursos e os docentes nas atividades 
      de ensino;

\item gerenciar os processos para transferências externas, internas e obtenção de novo 
     título;

\item coordenar o agendamento de ambientes do campus para utilização de aulas, reuniões 
      e eventos;

\item manter atualizados os Projetos Pedagógicos e enviá-los a Coordenadoria de Registro 
      e Controle Acadêmico;

\item manter o site institucional atualizado com as informações gerais do Ensino, 
      como Projetos Pedagógicos dos Cursos, Legislações, Horários de Aulas, entre outras;

\item desempenhar outras funções que, por sua natureza, lhe estejam correlatas ou lhe 
      tenham sido atribuídas.
\end{enumerate}
 
Art.~\ORD{6}{o} Ao(À) Diretor(a) de Ensino compete:

\begin{enumerate}
\renewcommand{\labelenumi}{\Roman{enumi}}

\item planejar, coordenar, supervisionar e avaliar as atividades dos setores e seções
      administrativas que compõe a Diretoria de Ensino;

\item assegurar o cumprimento da legislação educacional vigente e das normas institucionais 
      afetas à sua área de atuação;

\item propor medidas que visem à melhoria contínua e otimização dos métodos de trabalho;

\item elaborar um cronograma anual de ações internas prioritárias dos setores e seções que 
      compõe a diretoria, objetivando o bom andamento das atividades de ensino do campus;

\item propor aos órgãos competentes, a adoção de medidas necessárias à estruturação    
      curricular dos cursos em seus aspectos legais e formais, ao aperfeiçoamento da 
      administração acadêmica e à melhoria das condições materiais do ensino;

\item assessorar os coordenadores de cursos e os docentes nas atividades de ensino;

\item emitir parecer em processos que envolvam assuntos relativos à sua área de atuação;

\item participar das reuniões e discussões do Comitê de Ensino do IFMG;

\item planejar a capacitação e o aprimoramento profissional dos servidores lotados na   
      Diretoria de Ensino;

\item auxiliar o planejamento orçamentário anual do campus;

\item desempenhar outras funções que, por sua natureza, lhe estejam correlatas ou lhe  
      tenham sido atribuídas pelo Diretor-Geral.
\end{enumerate}

Art.~\ORD{7}{o} Competem aos servidores lotados diretamente na Diretoria de Ensino as 
seguintes atividades:

\begin{enumerate}
\renewcommand{\labelenumi}{\Roman{enumi}}

\item prestar atendimento presencial, e pelos meios de comunicação institucionais, à  
      comunidade acadêmica a fim de sanar dúvidas em relação às atividades desenvolvidas 
      pelo campus e, conforme o caso, realizar o encaminhamento ao setor responsável;

\item conduzir o processo de elaboração dos horários oficiais de aulas e de recuperação;

\item contribuir na elaboração do calendário acadêmico de referência do Campus;

\item conduzir semestralmente o Programa de Monitoria e Tutoria no que tange ao 
      levantamento de demanda, elaboração de editais, acompanhamento de inscrições e 
      avaliações por parte dos docentes, publicação de resultados, divulgação dos 
      horários, planilhas de pagamento e emissão de documentos;
      
\item conduzir os processos de transferências e obtenção de novo título para os cursos de           
      graduação, no que tange a elaboração de editais, recebimento de inscrições,  
      recebimento de recursos, avaliações por parte das coordenadores de curso, e 
      publicação dos resultados

\item gerenciar o uso dos ambientes do campus;

\item auxiliar a Pesquisa Institucional do IFMG no que concerne as informações e processos 
      do sistema e-MEC;

\item dar assistência ao corpo docente em relação aos materiais necessários para realização 
      das aulas

\item avaliar as alterações dos projetos pedagógicos dos cursos realizadas pelos órgãos          colegiados, verificando o cumprimento das diretrizes curriculares dos Cursos, das 
      normas institucionais e das demais legislações pertinentes;

\item arquivar as versões, manter atualizados os projetos pedagógicos e enviá-los ao Setor 
      de Registro e Controle Acadêmico

\item manter o site institucional atualizado com as informações gerais do ensino: projetos 
      pedagógicos dos cursos, horários das aulas, entre outras;

\item desempenhar outras funções que, por sua natureza, lhe estejam correlatas ou lhe  
      tenham sido atribuídas pelo Diretor(a) de Ensino.
\end{enumerate}

%%%%%%%%%%%%%%%%%%%%%%%%%%%%%%%%%%%%%%%%%%%%%%%%%%%%%%%%%%%%%%%%%%%%%%%%%%%%%%%%%%%%%%%%%%
%%%%%%            CAPÍTULO VI DAS COMPETÊNCIAS DOS SETORES E SECOES                 %%%%%% 
%%%%%%%%%%%%%%%%%%%%%%%%%%%%%%%%%%%%%%%%%%%%%%%%%%%%%%%%%%%%%%%%%%%%%%%%%%%%%%%%%%%%%%%%%%

\section{CAPÍTULO VI \\ DAS COMPETÊNCIAS DOS SETORES E SEÇÕES}
 
%%%%%%%%%%%%%%%%%%%%%%%%%%%%%%%%%%%%%%%%%%%%%%%%%%%%%%%%%%%%%%%%%%%%%%%%%%%%%%%%%%%%%%%%%%
%%%%%%                   SETOR DE REGISTRO E CONTROLE ACADÊMICO                     %%%%%% 
%%%%%%%%%%%%%%%%%%%%%%%%%%%%%%%%%%%%%%%%%%%%%%%%%%%%%%%%%%%%%%%%%%%%%%%%%%%%%%%%%%%%%%%%%%

\subsection{Seção I \\ Do Setor de Registro e Controle Acadêmico}

Art.~\ORD{8}{o} O Setor de Registro e Controle Acadêmico é um setor da Diretoria de Ensino 
ao qual compete:

\begin{enumerate}
\renewcommand{\labelenumi}{\Roman{enumi}}

\item coordenar as atividades de registro e documentação referentes a todos os níveis e 
      modalidades de cursos regulares ofertados pela instituição;

\item definir os processos internos, formulários e procedimentos a serem adotados em todos 
      os níveis e modalidades de ensino do Campus;

\item acompanhar e aplicar a legislação vigente sobre os registros acadêmicos;

\item supervisionar a alimentação de dados e a manutenção no sistema acadêmico;

\item realizar o planejamento dos eventos, a serem previstos em calendário acadêmico,  
      referentes aos períodos letivos para registro e manutenção das atividades acadêmicas 
      regulares dos discentes;

\item gerenciar os processos de solicitação de matrículas, trancamentos, segunda chamada de         
      atividades, aproveitamento de estudos, transferências, desligamentos, revisão de 
      provas, notas e frequência, entre outros pertinentes à vida acadêmica dos discentes;

\item alimentar/atualizar os dados cadastrais dos discentes no sistema de registro e 
      controle acadêmico, planilhas e relatórios internos e externos;

\item alimentar/atualizar o sistema acadêmico semestralmente com os dados necessários ao 
      início de cada período letivo;

\item Fazer a apuração dos resultados finais dos períodos letivos e informar a Comissão de 
      Formatura os alunos aptos à colação;

\item emitir toda a documentação referente à vida acadêmica do corpo discente e declarações 
      com as disciplinas ministradas pelos docentes.

\item realizar o envio de informações via e-mail e/ou SEI, solicitada pelos docentes, 
      coordenadorias, direção de ensino e geral e discentes do Campus;

\item arquivar os projetos pedagógicos dos cursos após parecer da Seção de Graduação e 
      Técnico, de forma a atender as solicitações e subsidiar as avaliações do MEC;

\item realizar as atividades de registro e arquivamento dos Planos de Ensino e dos Diários 
      e Planos de Aula, mantendo atualizado o sistema de controle acadêmico;

\item auxiliar a Pesquisa Institucional do IFMG no levantamento dos dados necessários para 
      alimentação dos sistemas do MEC, no preenchimento do Censo da Educação Superior e 
      da Educação Básica, na atualização do Sistema de Segurança Digital de Informações 
      Técnicas e Tecnológicas (SISTEC) e na validação dos dados da Plataforma Nilo Peçanha, 
      que subsidia a Matriz Orçamentária e os indicadores do TCU;

\item emitir relatório de turmas e ou cursos quando solicitados pelos coordenadores;

\item realizar o registro de certificados e diplomas de conclusão das habilitações em todos 
      os níveis e modalidades de cursos ofertados pela instituição;

\item prestar atendimentos ao público externo à Instituição referente 
      Certificação/Declaração de Proficiência do Ensino Médio com base no ENEM e demais 
      informações;

\item responsabilizar-se pela elaboração de relatórios de desempenho dos serviços de sua   
      responsabilidade;

\item auxiliar na elaboração do orçamento e do planejamento anual do Campus;

\item desempenhar outras funções que, por sua natureza, lhe estejam correlatas ou lhe   
      tenham sido atribuídas pelo Diretor(a) de Ensino.
\end{enumerate}


Art.~\ORD{9}{o} Ao Chefe do Setor de Registro e Controle Acadêmico compete:

\begin{enumerate}
\renewcommand{\labelenumi}{\Roman{enumi}}

\item coordenar o trabalho dos servidores lotados no setor, objetivando a execução do 
      cronograma de ações prioritárias sob responsabilidade do setor e a otimização dos 
      recursos humanos;
	
\item coordenar as atividades de registro e documentação referentes a todos os níveis e 
      modalidades de cursos regulares ofertados pela instituição;
	
\item definir os processos internos, formulários e procedimentos a serem adotados em todos 
      os níveis e modalidades de ensino do Campus;
	
\item acompanhar e aplicar a legislação vigente sobre os registros acadêmicos;
	
\item supervisionar a alimentação de dados e a manutenção no sistema acadêmico;
	
\item realizar o planejamento dos eventos, a serem previstos em calendário acadêmico, 
      referentes aos períodos letivos para registro e manutenção das atividades acadêmicas 
      regulares dos discentes;
	
\item gerenciar os processos de solicitação de matrículas, trancamentos, segunda chamada de       atividades, aproveitamento de estudos, transferências, desligamentos, revisão de 
      provas, notas e frequência, entre outros pertinentes à vida acadêmica dos discentes;
	
\item fazer a apuração dos resultados finais dos períodos letivos e informar a Comissão de 
      Formatura os alunos aptos à colação;

\item realizar o registro de certificados e diplomas de conclusão das habilitações em todos
      os níveis e modalidades de cursos ofertados pela instituição;

\item auxiliar a Pesquisa Institucional do IFMG no levantamento dos dados necessários para 
      alimentação dos sistemas do MEC, no preenchimento do Censo da Educação Superior e 
      da Educação Básica, na atualização do Sistema de Segurança Digital de Informações 
      Técnicas e Tecnológicas (SISTEC) e na validação dos dados da Plataforma Nilo 
      Peçanha, que subsidia a Matriz Orçamentária e os indicadores do TCU;

\item auxiliar o planejamento orçamentário anual do Campus;

\item desempenhar outras funções que, por sua natureza, lhe estejam correlatas ou lhe   
      tenham sido atribuídas pelo Diretor(a) de Ensino.
\end{enumerate}

Art.~10 Competem aos servidores lotados no Setor de Registro e Controle Acadêmico as 
        seguintes atividades:

\begin{enumerate}
\renewcommand{\labelenumi}{\Roman{enumi}}

\item prestar atendimento presencial, e pelos meios de comunicação institucionais, à 
      comunidade acadêmica a fim de  sanar dúvidas em relação às atividades desenvolvidas 
      pela campus e, conforme o caso, realizar o encaminhamento ao setor responsável;

\item prestar atendimentos ao público externo à Instituição referente à  
      Certificação/Declaração de Proficiência do Ensino Médio com base no ENEM e demais 
      informações;

\item disponibilizar requerimentos aos discentes no que se refere aos processos de     
      solicitação de matrículas, trancamentos, segunda chamada de atividades, aproveitamento 
      de estudos, transferências, desligamentos, revisão de provas, notas e frequência, 
      entre outros pertinentes à vida acadêmica dos discentes, dentro dos períodos 
      definidos em calendário acadêmico; 

\item alimentar/atualizar os dados cadastrais dos discentes no sistema de registro e 
      controle acadêmico, planilhas e relatórios internos e externos;

\item alimentar/atualizar o sistema acadêmico semestralmente com os dados necessários ao  
      início de cada período letivo;

\item realizar o envio de informações via e-mail e/ou SEI, solicitada pelos docentes, 
      coordenadorias, direção de ensino e geral e discentes do Campus;

\item enviar documentos dos discentes às Coordenações de Cursos para análise;

\item registrar em ata os processos de colação de grau;

\item realizar as atividades de registro e arquivamento dos Planos de Ensino e dos Diários 
      e Planos de Aula, mantendo atualizado o sistema de controle acadêmico;
      
\item emitir relatório de turmas e ou cursos quando solicitados pelos coordenadores;

\item auxiliar a Pesquisa Institucional do IFMG no levantamento dos dados necessários 
      para alimentação dos sistemas do MEC, no preenchimento do Censo da Educação Superior 
      e da Educação Básica, na atualização do Sistema de Segurança Digital de Informações 
      Técnicas e Tecnológicas (SISTEC) e na validação dos dados da Plataforma Nilo Peçanha, 
      que subsidia a Matriz Orçamentária e os indicadores do TCU;

\item arquivar os projetos pedagógicos dos cursos após parecer da Seção de Graduação e 
      Técnico, de forma a atender as solicitações e subsidiar as avaliações do MEC;

\item responsabilizar-se pela guarda de documentos inerentes ao setor, considerando os 
      prazos legais;

\item desempenhar outras funções que, por sua natureza, lhe estejam correlatas ou lhe 
      tenham sido atribuídas pelo Chefe do Setor de Registro e Controle Acadêmico.

\end{enumerate}

%%%%%%%%%%%%%%%%%%%%%%%%%%%%%%%%%%%%%%%%%%%%%%%%%%%%%%%%%%%%%%%%%%%%%%%%%%%%%%%%%%%%%%%%%%
%%%%%%                     SEÇÃO DE PLANEJAMENTO DE ENSINO DE GRADUAÇÃO             %%%%%%
%%%%%%%%%%%%%%%%%%%%%%%%%%%%%%%%%%%%%%%%%%%%%%%%%%%%%%%%%%%%%%%%%%%%%%%%%%%%%%%%%%%%%%%%%%

\subsection{Seção II \\ Da Seção de Planejamento de Ensino de Graduação}

Art.~11 À Seção de Planejamento de Ensino de Graduação é uma seção da Diretoria de Ensino 
à qual compete:

\begin{enumerate}
\renewcommand{\labelenumi}{\Roman{enumi}}

\item supervisionar e propor ações aos coordenadores dos cursos de graduação quanto aos 
      ciclos avaliativos do Exame Nacional de Desempenho dos Estudantes;
      
\item supervisionar o cumprimento do Regulamento de Ensino dos Cursos de Graduação pelos 
      coordenadores de cursos  e o funcionamento dos Colegiados e Núcleos Docentes 
      Estruturantes (NDE);
      
\item promover, juntamente com os coordenadores dos cursos de graduação, reflexões e 
      medidas que visem à melhoria contínua da qualidade dos cursos de graduação;
      
\item supervisionar as ações dos coordenadores dos cursos de graduação quanto aos processos 
      de avaliação abertos no Sistema de Regulação e Avaliação do Ensino Superior (e-MEC);
      
\item acompanhar e avaliar as alterações dos projetos pedagógicos dos cursos realizadas 
      pelos órgãos colegiados, verificando o cumprimento das diretrizes curriculares dos 
      Cursos, das normas institucionais e das demais legislações pertinentes;

\item supervisionar o processo de transição entre coordenadores dos cursos de graduação;

\item auxiliar na elaboração dos calendários acadêmicos de referência do campus;

\item desempenhar outras funções que, por sua natureza, lhe estejam correlatas ou lhe 
      tenham sido atribuídas pelo(a) Diretor(a) de Ensino.
\end{enumerate}

Parágrafo-único. As atribuições da Seção de Planejamento de Ensino de Graduação serão 
desempenhadas pelo Chefe da Seção, o qual será designado pelo Diretor-Geral, ouvido 
o Diretor(a) de Ensino.

%%%%%%%%%%%%%%%%%%%%%%%%%%%%%%%%%%%%%%%%%%%%%%%%%%%%%%%%%%%%%%%%%%%%%%%%%%%%%%%%%%%%%%%%%%
%%%%%%                     SEÇÃO DE PLANEJAMENTO DE ENSINO                          %%%%%%
%%%%%%%%%%%%%%%%%%%%%%%%%%%%%%%%%%%%%%%%%%%%%%%%%%%%%%%%%%%%%%%%%%%%%%%%%%%%%%%%%%%%%%%%%%

\subsection{Seção III \\ Da Seção de Planejamento de Ensino}

Art.~12 À Seção de Planejamento de Ensino é uma seção da Diretoria de Ensino à qual compete:

\begin{enumerate}
\renewcommand{\labelenumi}{\Roman{enumi}}

\item supervisionar o cumprimento do Regulamento de Ensino dos Cursos de Educação 
      Profissional Técnica de Nível Médio pelos coordenadores de cursos e o funcionamento 
      dos colegiados dos cursos técnicos;

\item promover, juntamente com os coordenadores dos cursos técnicos, reflexões e medidas
      que visem à melhoria dos currículos dos cursos técnicos;
      
\item acompanhar e avaliar as alterações dos projetos pedagógicos dos cursos realizadas 
      pelos órgãos colegiados, verificando o cumprimento das diretrizes curriculares dos 
      Cursos, das normas institucionais e das demais legislações pertinentes;
        
\item supervisionar o processo de transição entre coordenadores dos cursos de graduação;

\item auxiliar na elaboração dos calendários acadêmicos de referência do campus;

\item desempenhar outras funções que, por sua natureza, lhe estejam correlatas ou lhe 
      tenham sido atribuídas pelo(a) Diretor(a) de Ensino.
\end{enumerate}

Parágrafo-único. As atribuições da Seção de Planejamento de Ensino de Graduação serão 
desempenhadas pelo Chefe da Seção, o qual será designado pelo Diretor-Geral, ouvido 
o Diretor(a) de Ensino.

%%%%%%%%%%%%%%%%%%%%%%%%%%%%%%%%%%%%%%%%%%%%%%%%%%%%%%%%%%%%%%%%%%%%%%%%%%%%%%%%%%%%%%%%%%
%%%%%%                          SEÇÃO PEDAGÓGICA                                    %%%%%%
%%%%%%%%%%%%%%%%%%%%%%%%%%%%%%%%%%%%%%%%%%%%%%%%%%%%%%%%%%%%%%%%%%%%%%%%%%%%%%%%%%%%%%%%%%

\subsection{Seção IV \\ Da Seção Pedagógica}

Art.~13 A Seção Pedagógica é uma seção da Diretoria de Ensino à qual compete:

\begin{enumerate}
\renewcommand{\labelenumi}{\Roman{enumi}}

\item prestar atendimento multidisciplinar aos alunos e respectivos pais ou responsáveis,  
      considerando os aspectos pedagógicos, psicológicos, de saúde, sociais e disciplinares 
      e produzir relatórios dos atendimentos.  
     
\item executar o Programa de Assistência Estudantil no âmbito do Campus, considerando o 
      disposto na Política de Assistência Estudantil do IFMG vigente;
      
\item gerenciar o processo de aquisição e distribuição dos livros didáticos dos Cursos 
      Técnicos Integrados ao Ensino Médio;

\item executar o processo de distribuição dos alimentos adquiridos por meio do Programa 
      Nacional de Alimentação Escolar (PNAE) aos alunos dos cursos técnicos;
      
\item organizar e conduzir as reuniões de Conselho de Classe;

\item comunicar aos pais ou responsáveis e ao Conselho Tutelar, a porcentagem de faltas 
      para os alunos dos cursos técnicos integrados, menores, conforme a legislação 
      vigente;

\item organizar e conduzir as reuniões iniciais de pais de alunos ingressantes;
    
\item apoiar as ações da Comissão Disciplinar Discente (CDD) no que concerne à notificação 
      dos pais e responsáveis, à aplicação de advertências orais e no acompanhamento de 
      medidas socioeducativas;
      
\item realizar estudos sobre taxas de conclusão, evasão e retenção escolar;
 
\item desempenhar outras funções que, por sua natureza, lhe estejam correlatas ou lhe 
      tenham sido atribuídas pelo(a) Diretor(a) de Ensino.
\end{enumerate}


Art.~14 Ao Chefe da Seção Pedagógica compete:

\begin{enumerate}
\renewcommand{\labelenumi}{\Roman{enumi}}

\item coordenar e supervisionar o trabalho dos servidores lotados na seção, objetivando a 
      execução do cronograma de ações prioritárias sob responsabilidade da seção e a 
      otimização dos recursos humanos;

\item auxiliar as coordenadorias dos cursos em questões pedagógicas;

\item auxiliar na elaboração dos calendários acadêmicos de referência do Campus;

\item desempenhar outras funções que, por sua natureza, lhe estejam correlatas ou lhe 
      tenham sido atribuídas pelo(a) Diretor(a) de Ensino.
\end{enumerate}


Art.~15 Competem aos servidores lotados na Seção Pedagógica as seguintes atividades:

\begin{enumerate}
\renewcommand{\labelenumi}{\Roman{enumi}}

\item prestar atendimento presencial, e pelos meios de comunicação institucionais, à 
      comunidade acadêmica a fim de sanar dúvidas em relação às atividades desenvolvidas 
      pela campus e, conforme o caso, realizar o encaminhamento ao setor responsável;

\item atender pais e alunos dos Cursos Técnicos Integrados em relação ao desejo de 
      solicitar transferência e promover reflexão sobre a decisão, emitindo quando 
      definido, declaração de nada consta para iniciar o processo formal junto ao 
      setor de Registro e Controle Acadêmico;

\item receber e entregar os livros didáticos (PNLD) aos alunos dos Cursos Técnicos 
      Integrados;

\item orientar sobre o agendamento de horário para atendimentos especializados, ou 
      mediação nos casos em que não for viável o agendamento;

\item dar assistência e orientação aos discentes no aspecto da disciplina, lazer, 
      segurança, saúde, pontualidade, higiene, dentro das dependências escolares;

\item organizar e distribuir os alimentos do Programa Nacional de Alimentação Escolar 
      (PNAE) aos alunos dos Cursos Técnicos;

\item dar assistência ao corpo docente, quando demandado, nas questões didático-pedagógicas 
      com os materiais necessários a execução de suas atividades;

\item elaborar planilhas com os dados e fotos dos alunos ingressantes dos Cursos Técnicos;

\item orientar os discentes em relação aos atos infracionais no ambiente escolar e apoiar 
      às ações da Comissão Disciplinar Discente (CDD), como notificação, aplicação de 
      advertências orais e acompanhamento de medidas socioeducativas;

\item comunicar aos pais e Conselho Tutelar, o alcance da porcentagem de faltas para os 
      alunos dos Cursos Técnicos Integrados, menores, exigida na legislação vigente;

\item prestar atendimento multidisciplinar aos alunos e responsáveis considerando os 
      aspectos pedagógicos, psicológicos e sociais;

\item realizar orientação pedagógica, psicológica e social aos docentes e discentes, 
      quando demandado;

\item acompanhar os alunos em situação de regime excepcional de estudos encaminhados 
      pelo Setor de Registro e Controle Acadêmico;

\item organizar e participar da Reunião inicial do período letivo de Pais de alunos 
      ingressantes;

\item participar e organizar os encaminhamentos aos responsáveis por alunos em situações 
      críticas detectadas nas reuniões dos conselhos de classe;

\item estudar medidas que visem melhorar os processos pedagógicos e apresentar sugestões a 
      chefia imediata;

\item elaborar apresentação com orientações para os alunos ingressantes;

\item desenvolver atividades voltadas para o favorecimento da permanência do estudante, 
      minimizando a evasão, promovendo a melhoria do desempenho acadêmico e colaborando 
      para a inclusão social;

\item conduzir o processo de aquisição e distribuição dos livros didáticos dos Cursos 
      Técnicos Integrados ao Ensino Médio;

\item executar as ações ligadas ao Programa de Assistência Estudantil, no que tange a 
      divulgação dos  Editais, recebimentos de documentos, elaboração e ajuste mensal da 
      planilha de pagamento dos bolsistas e disponibilização para o pagamento e demais 
      ações previstas na Política de Assistência Estudantil vigenge no IFMG;

\item acompanhar os alunos assistidos pelo NAPNEE (Núcleo de Atendimento às Pessoas com 
      Necessidades Educacionais Específicas);

\item efetuar comunicação entre surdos e ouvintes, surdos e surdos, surdos e surdos-cegos, 
      surdos-cegos e ouvintes, por meio da Língua Brasileira de Sinais para a língua oral 
      (portuguesa) e vice-versa;
      
\item interpretar, em Língua Brasileira de Sinais - Língua Portuguesa, as atividades 
      didático-pedagógicas e culturais desenvolvidas na instituição de forma a viabilizar 
      o acesso aos conteúdos curriculares;

\item responsabilizar-se pela guarda de documentos inerentes a sua função, considerando 
      os prazos legais;

\item desempenhar outras funções que, por sua natureza, lhe estejam correlatas ou lhe 
      tenham sido atribuídas pelo(a) Chefe da Seção Pedagógica.

\end{enumerate}

%%%%%%%%%%%%%%%%%%%%%%%%%%%%%%%%%%%%%%%%%%%%%%%%%%%%%%%%%%%%%%%%%%%%%%%%%%%%%%%%%%%%%%%%%%
%%%%%%                        SEÇÃO DE LABORATÓRIOS                                 %%%%%%
%%%%%%%%%%%%%%%%%%%%%%%%%%%%%%%%%%%%%%%%%%%%%%%%%%%%%%%%%%%%%%%%%%%%%%%%%%%%%%%%%%%%%%%%%%

\subsection{Seção V \\ Da Seção de Laboratórios}

Art.~16 A Seção de Laboratórios é uma seção da Diretoria de Ensino à qual compete:

\begin{enumerate}
\renewcommand{\labelenumi}{\Roman{enumi}}

\item coordenar e supervisionar a utilização dos laboratórios dedicados ao ensino e o 
      cumprimento das normas disciplinares e de segurança;
      
\item planejar e solicitar a compra de equipamentos e materiais para a utilização adequada 
      dos laboratórios, em parceria com os coordenadores de cursos e docentes;

\item fornecer suporte especializado às aulas práticas nos laboratórios dedicados ao 
      ensino;

\item propor e manter atualizadas normas de uso dos laboratórios e dar publicidade à 
      comunidade acadêmica;

\item desempenhar outras funções que, por sua natureza, lhe estejam correlatas ou lhe 
      tenham sido atribuídas pelo(a) Diretor(a) de Ensino.
\end{enumerate}


Art.~17 Ao Chefe da Seção de Laboratórios compete:

\begin{enumerate}
\renewcommand{\labelenumi}{\Roman{enumi}}

\item planejar e coordenar o trabalho dos técnicos de laboratório, visando o bom andamento 
      das atividades acadêmicas desenvolvidas nos laboratórios e a otimização dos recursos 
      humanos.

\item planejar anualmente a compra de equipamentos e materiais para a utilização adequada 
      dos laboratórios, em parceria com os coordenadores de cursos e docentes;

\item Participar do processo de elaboração do horário das aulas e enviar as coordenadorias 
      dos cursos sugestões de alocação de disciplinas e apontando restrições técnicas que 
      visem otimizar a utilização dos recursos materiais e humanos nas aulas de laboratórios

\item desempenhar outras funções que, por sua natureza, lhe estejam correlatas ou lhe 
      tenham sido atribuídas pelo(a) Diretor(a) de Ensino.
\end{enumerate}


Art.~18 Competem aos servidores lotados na Seção de Laboratórios:

\begin{enumerate}
\renewcommand{\labelenumi}{\Roman{enumi}}

\item fornecer suporte especializado às aulas práticas nos laboratórios dedicados ao 
      ensino;

\item zelar pelos laboratórios específicos de ensino e pelo o cumprimento das normas 
      de segurança e disciplinares durante o uso dos laboratórios;

\item preparar os recursos necessários para uso dos laboratórios antes do início de cada 
      período letivo; 

\item propor e manter atualizadas normas de uso dos laboratórios e dar publicidade à 
      comunidade acadêmica;

\item desempenhar outras funções que, por sua natureza, lhe estejam correlatas ou lhe 
      tenham sido atribuídas pelo(a) Chefe da Seção de Laboratórios
\end{enumerate}

%%%%%%%%%%%%%%%%%%%%%%%%%%%%%%%%%%%%%%%%%%%%%%%%%%%%%%%%%%%%%%%%%%%%%%%%%%%%%%%%%%%%%%%%%%
%%%%%%                      SEÇÃO DE ASSUNTOS ESTUDANTIS                            %%%%%%
%%%%%%%%%%%%%%%%%%%%%%%%%%%%%%%%%%%%%%%%%%%%%%%%%%%%%%%%%%%%%%%%%%%%%%%%%%%%%%%%%%%%%%%%%%

\subsection{Seção V \\ Da Seção de Assuntos Estudantis}

Art.~19 A Seção de Assuntos Estudantis é uma seção da Diretoria de Ensino à qual compete:

\begin{enumerate}
\renewcommand{\labelenumi}{\Roman{enumi}}

\item implementar a política biblioteconômica do IFMG;

\item promover a guarda, conservação e restauração do acervo bibliográfico e de outros 
      materiais de uso didático sob sua responsabilidade;

\item assistir e orientar os usuários da biblioteca quanto à utilização adequada do acervo 
      bibliográfico físico e virtual e de outras fontes de informações;

\item promover a melhoria contínua dos serviços disponibilizados pelo da biblioteca;

\item responsabilizar-se pela guarda de documentos inerentes ao setor

\item desempenhar outras funções que, por sua natureza, lhe estejam correlatas ou lhe 
      tenham sido atribuídas pelo(a) Diretor(a) de Ensino.
\end{enumerate}


Art.~20 Ao Chefe da Seção de Assuntos Estudantis compete:

\begin{enumerate}
\renewcommand{\labelenumi}{\Roman{enumi}}

\item planejar e coordenar o trabalho dos servidores lotados na seção, visando o 
      cumprimento das ações prioritárias sob a responsabilidade do setor, a oferta 
      adequada de serviços e a otimização dos recursos humanos;

\item realizar estudos e propor melhorias para o dimensionamento de equipamentos, dos 
      recursos humanos e do lay-out dos espaços disponíveis;

\item contribuir e participar das ações desenvolvidas pela Rede de Bibliotecas do IFMG; 

\item contribuir para a realização do inventário anual do setor;

\item desempenhar outras funções que, por sua natureza, lhe estejam correlatas ou lhe 
      tenham sido atribuídas pelo(a) Diretor(a) de Ensino.
\end{enumerate}


Art.~21 Competem aos servidores lotados na Seção de Assuntos Estudantis as seguintes 
atividades:

\begin{enumerate}
\renewcommand{\labelenumi}{\Roman{enumi}}

\item executar os procedimentos de empréstimo, devolução e renovação do empréstimo de 
      livros;

\item orientar o usuário quanto ao acervo e localização do título;

\item orientar, treinar, incluir novos usuários em plataformas digitais de pesquisa e 
      conteúdo.  

\item emitir Declaração de nada consta a toda a comunidade acadêmica;

\item receber doações de novos exemplares, análise do material, catalogação, etiquetagem e 
      ou descarte;

\item receber e disponibilizar para consulta os trabalhos de conclusão de curso nas versões 
      física e/ou digital;

\item confeccionar Ficha Catalográfica dos trabalhos de conclusão de curso com mais de 60 
      folhas, observando o prazo de entrega;

\item orientar e apoiar as atualizações dos Projetos Pedagógicos dos Cursos em relação às 
      bibliografias básica e complementar;

\item organizar o espaço da Biblioteca e as salas de estudos disponíveis;

\item alfabetar as estantes de livros para a manutenção da organização e apresentação do 
      acervo;

\item realizar Inventário anual do acervo;

\item atuar no tratamento, recuperação e disseminação da informação e executar atividades 
      especializadas relacionadas à rotina da biblioteca, quer no atendimento ao usuário, 
      quer na administração do acervo ou na manutenção do bancos de dados;

\item colaborar no controle e na conservação de equipamentos;

\item realizar a manutenção do acervo;

\item participar de treinamentos e programas de atualização;

\item participar, conforme a política interna da Instituição, de projetos, cursos, eventos,       convênios e programas de ensino, pesquisa e extensão;

\item desempenhar outras funções que, por sua natureza, lhe estejam correlatas ou lhe 
      tenham sido atribuídas pelo(a) Chefe da Seção de Assuntos Estudantis.
\end{enumerate}

%%%%%%%%%%%%%%%%%%%%%%%%%%%%%%%%%%%%%%%%%%%%%%%%%%%%%%%%%%%%%%%%%%%%%%%%%%%%%%%%%%%%%%%%%%
%%%%%%              CAPÍTULO VII DAS DISPOSIÇÕES GERAIS E TRANSITÓRIAS              %%%%%%
%%%%%%%%%%%%%%%%%%%%%%%%%%%%%%%%%%%%%%%%%%%%%%%%%%%%%%%%%%%%%%%%%%%%%%%%%%%%%%%%%%%%%%%%%%

\section{CAPÍTULO VII \\ DAS DISPOSIÇÕES GERAIS E TRANSITÓRIAS}

Art.~22 Os procedimentos operacionais necessários ao bom funcionamento do ensino no campus, 
a serem definidos pelo(a) Diretor(a) de Ensino, deverão ser documentados e formalizados 
por portaria específica.  

Art.~23 As atribuições básicas e especializadas, dos respectivos cargos, dos servidores 
lotados nos órgãos que compõe a Diretoria de Ensino serão descritas em portaria específica.

Art.~24 Os casos omissos serão resolvidos pelo(a) Diretor(a) de Ensino, ouvido o Setor ou Seção pertinente.

Art.~25 Esta portaria deverá ser publicada no Boletim Eletrônico de Serviços do IFMG.

\end{document}
